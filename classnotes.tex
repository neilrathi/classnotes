\documentclass[11pt]{article}
\usepackage[utf8]{inputenc}
\usepackage[transitional, mathfont]{classnotes}

\title{Using the Classnotes Package}
\author{Neil Rathi}

\begin{document}\thispagestyle{empty}

\maketitle

\tableofcontents

\newpage

\section{Introduction}
The \texttt{classnotes.sty} package is a package designed mostly for taking lecture notes, as the name suggests. It comes with multiple pre-loaded packages useful for typesetting notes in various subjects, including math, chemistry, linguistics, and physics. I created this package mostly because I realized that my preambles were getting way too long for a normal document, and decided to standardize across domains so that I could use just one package when taking notes.

It's simple to create a document using the package, and the package does most of the important formatting automatically, like the title, table of contents, and footer. You can optionally pass the \texttt{human}, \texttt{gara}, \texttt{transitional}, \texttt{dido}, \texttt{mech}, \texttt{sans}, or \texttt{cms} options. The first five correspond to their Vox-ATypI classifications (Coelacanth, Garamond, Libertinus\footnote{I prefer Crimson, but I needed \texttt{scshape}. If you have Clara or Spectral installed, that's another option.}, \TeX{} Gyre Schola, and Computer Concrete, respectively), the second to last is Latin Modern Sans, and the last is Computer Modern Sans. Otherwise, the default font is Computer Modern Serif. This guide uses the \verb|transitional| option. You can also use the \verb|mathfont| option to change the font in math mode from default CMS to match the style of the selected font (I used it here to demonstrate, but I recommend not).

\subsection{About this Guide}
The purpose of this document isn't to give a comprehensive list of every feature in the package. Instead, it provides an overview of some of the most fundamental pieces of the package, and also shows what the package actually looks like in use.

I also used this guide while I was editing and creating the package to ensure that everything was working smoothly. To see the entire list of features, go to the \texttt{README.md} file in the github repository (\url{https://github.com/neilrathi/classnotes}).

\section{Mathematics}
The package has multiple pre-loaded packages for math, including \texttt{amsmath}, \texttt{amssymb}, \texttt{amsthm}, and \texttt{mathtools}. In addition, there are a few important and useful shorthand commands.

\subsection{Theorems}
The most prominent feature of the package is in the newly defined environments. The package uses \texttt{mdframed} and \texttt{thmtools} to create theorem boxes that clearly stand out from the surrounding text, and draw the readers attention. In the standard package, all of the boxes are gray with a thin border, and the \texttt{color} option makes the boxes colorful and even more prominent. As an example, we can write the following theorem using \verb|\begin{thrm}| and \verb|\end{thrm}|.
\begin{thrm}[Fermat's Last Theorem]
	There are no three integers $x,y,z > 0$ such that
	\[x^n + y^n = z^n\]
	for any $n > 2 \in \ZZ$.
\end{thrm}
We can then follow any theorem with a \verb|proof| (which is really the same as the \texttt{amsthm} one, but with a \verb|\blacksquare|).

\begin{proof}
	We show that the equation is satisfied for $n = 1$ and $n = 2$. For the first case, note that this reduces to
	\[x + y = z,\]
	which obviously holds. The second case is again trivial, with the equation being
	\[x^2 + y^2 = z^2.\]
	This is, of course, the Pythagorean theorem. The proof of $n > 2$ is trivial and is left as an exercise to the reader. See \cite{wiles1995} for more.
\end{proof}

There are also other useful environments which often appear during lectures or classes. These environments share the same visual aesthetic as the theorem environment, but with different header styles. For example (pun intended), we have the \verb|example| environment, which mirrors the theorem environment.
\begin{example}[2005 Putnam A5]
	Evaluate the integral
	\[\int_0^1 \frac{\ln(x+1)}{x^2+1} \dd x\]
\end{example}
Sometimes, an example like the one above may be very challenging, so the instructor or textbook will give a \verb|hint|, like the one below.
\begin{hint}
	Prove that for a function $f: \RR \to \RR$,
	\[\int_0^a f(x) \dd x = \int_0^a f(a-x) \dd x\]
	where $a$ is a real number.
\end{hint}
Of course, examples don't necessarily have \textit{proofs}, they have solutions. The package provides a \verb|solution| environment identical to the \verb|proof| environment, but with the italicized word changed.
\begin{solution}
	We take the hint as a given and leave the proof to the reader. Substituting $x$ with $\tant$, we have
	\begin{equation}
		I = \int_0^{\pi/4} \ln(1+\tant)\dd\theta.
	\end{equation}
	Using the identity provided in the hint, we have
	\begin{align}
		I &= \int_0^{\pi/4} \ln\left(1+\tan\left(\frac{\pi}{4}-\theta\right)\right)	\dd\theta \\
		&= \int_0^{\pi/4} \ln\left(\frac{2}{1+\tant}\right)\dd\theta.
	\end{align}
	Adding the equations (2.1) and (2.3) together yields
	\begin{align}
		2I &= \int_0^{\pi/4} \ln(2)\dd\theta \\
		&= \frac{\pi\ln2}{4}
	\end{align}
	so $I = \boxed{\dfrac{\pi\ln2}{8}}$.
\end{solution}

The final important boxed environment is the \verb|defn| environment, used for definitions, which is in the same style as the \verb|example| environment.
\begin{defn}[Entropy]
	The \textit{entropy}, or average information, of a discrete random variable $X$ with outcomes $x_1...x_n$ each which have probability $P(x_1)...P(x_n)$, respectively, is
	\[H[X] = -\sum_{i=1}^n P(x_i)\log P(x_i),\]
	it's average information \cite{shannon1948}.
\end{defn}
In total, there are three different classes of theorem environment, \verb|thm|, \verb|defn|, and \verb|comment|. Each of the environments belongs to one of these three stylistic groups. The last main environment is the \verb|problem| environment, ideal for end-of-chapter problems. Unlike the above environments, these are less clearly separated from the text.
\begin{problem}
	Evaluate the expression $1 + 1$
\end{problem}
\begin{problem}[AMC 12 \#17]
	Seven cubes, whose volumes are 1, 8, 27, 64, 125, 216, and 343 cubic units, are stacked vertically to form a tower in which the volumes of the cubes decrease from bottom to top. Except for the bottom cube, the bottom face of each cube lies completely on top of the cube below it. What is the total surface area of the tower (including the bottom) in square units?
\end{problem}
\begin{problem}
	Prove whether or not there are infinitely many primes $p$ such that $p + 2$ or $p-2$ is prime.
\end{problem}

\subsection{Shorthand Commands}
The other feature of the mathematics section is simple shorthand commands for common functions which are either not defined or take too long to type quickly.

The \texttt{amsfont} package provides various useful fonts for math, including the \verb|\mathcal{}| font. The \texttt{classnotes} package creates the shorthand \verb|\mc| for \verb|\mathcal{}|, so that \verb|$\mcP$| becomes $\mcP$. There's also the blackboard bold \verb|\mathbb{}| font, which is encoded as \verb|$\ZZ$| for $\ZZ$.
Typing \verb|\left\lfloor x\right\rfloor| every time I wrote floor operations proved to be annoying, so the command \verb|\floor{x}| writes
\[\floor{\sum_{i=1}^{\infty} \frac{1}{i^2}} = 0.\]
The same applies for \verb|\ceil{x}|, \verb|\p{x}| (parenthesis), and for \verb|\bvec{x}| (bracketed vectors). The delimiters are automatically adjusted for size.

For trigonometry, a few helpful commands are provided. The \verb|\inv| command allows you to write \verb|\sin\inv| instead of \verb|\sin^{-1}|. For \verb|\sin\theta| and \verb|\sin x|, there are additionally the \verb|\sint| and \verb|\sinx| commands, which output $\sint$ and $\sinx$ (this is also enabled for the other 5 trig functions). Additionally, inverse trig functions of $\theta$ and $x$ have \verb|\sinit| and \verb|\sinix|. Finally, the much-needed \verb|\cis| operator is provided.

For mathematical logic, some shorthands are provided such as \verb|\is| instead of \verb|\equiv| ($\is$), \verb|\isE| instead of \verb|\exists| ($\isE$), and \verb|\xor| instead of \verb|\oplus| ($\xor$). The package has things like \verb|\Hom|, \verb|\Gal|, and \verb|\Aut| for algebra. The command \verb|\modop{a}{b}{c}| outputs $\modop{a}{b}{c}$. Various summations are also provided, like \verb|\cycsum|.

\subsection{Computer Science}
The pre-loaded packages for computer science are \texttt{algorithm2e}, \texttt{listings}, and \texttt{stmaryrd}. The basic shorthand commands are \verb|\defas|, which is the same as \verb|\coloneqq| ($\defas$), \verb|\setto|, a leftwards arrow ($\setto$), and \verb|\bigo|, a \texttt{mathcal} O for big-O notation ($\bigo$).

Another \texttt{thm} class environment is the \texttt{algo} environment, as seen below (using algorithm2e to typeset the algorithm itself)
\begin{algo}[Euclidean Algorithm]
	Given two nonnegative integers, recursively finds their greatest common divisor.
	
	\begin{algorithm}[H]
    	Euclid$(a,b)$:
    	
    	\eIf{$b=0$}
      	{
        	return $a$
      	}
      	{
        	return Euclid$(b,a\mod b)$
      	}	
	\end{algorithm}
\end{algo}
In addition to pseudocode algorithms, the package comes with three predefined styles for listings using \texttt{listings} (Python, Java, and C/C++). You can use \verb|\lststyle{pystyle}| to define the style, as below.

\lststyle{pystyle}
\begin{lstlisting}[caption=Extended Euclidean Algorithm in Python]
def euclid(a, b):  
    # base  
    if a == 0 :   
        return b, 0, 1
    # recursive step
    gcd, x1, y1 = euclid(b%a, a)
    x = y1 - (b//a) * x1  
    y = x1  
    return gcd, x, y 
\end{lstlisting}
Thankfully, \texttt{stmaryrd} provides most symbols for theoretical CS, so there aren't any other redefined symbols.

\section{Linguistics}
The package pre-loads the \texttt{tipa}, \texttt{phonrule}, \texttt{ot-tableau}, \texttt{tikz-dependency}, \texttt{forest}, and \texttt{linguex}. I chose to use \texttt{linguex} rather than \texttt{gb4e} because gb4e runs into an error with math mode.

\subsection{Phonetics and Phonology}
With tipa, I've included a few extra commands. The \verb|\ipa{}| command is a replacement for \verb|\textipa{}|. You can use \verb|\up| instead of \verb|\super| for superscripts in an IPA environment. For example, we can say \verb|\ipa{/DIs Iz t\up haIpt In "leItEk/}| to say \ipa{/DIs Iz t\up haIpt In "leItEk/}.

I've also created basic phonetic rule environments not provided by phonrule. The command \verb|\phonss{p}{t}{k}| outputs \phonss{p}{t}{k}, while the \verb|\phonsb{p}{t}{k}| command outputs \phonsb{p}{t}{k}. In general, \verb|s| means ``slashes" and \verb|b| means ``brackets". I also redefined the \verb|\ips{}| and \verb|\ipb{}| commands from \texttt{ot-tableau} using \texttt{tipa}:

\begin{center}
	\begin{tableau}{c|c|c|c|c|c}
	\inp{\ips{t-n-ak-ol}} \const{Ident-IO} \const{*[+low]} \const{*[+round]} \const{*[-back]} \const{Cor-high} \const{*[+high]}
	\cand{t@.na.g0l} \vio{} \vio{*} \vio{*} \vio{} \vio{**!} \vio{}
	\cand[\Optimal]{t1.na.g0l} \vio{} \vio{*} \vio{*} \vio{} \vio{*} \vio{*}
	\cand{ti.na.g0l} \vio{} \vio{*} \vio{*} \vio{*!} \vio{*} \vio{*}
	\cand{tu.na.g0l} \vio{} \vio{*} \vio{**!} \vio{} \vio{*} \vio{*}
	\cand{ta.na.g0l} \vio{} \vio{**!} \vio{*} \vio{} \vio{**} \vio{}
	\cand{t1.n1.g@l} \vio{*!*} \vio{} \vio{} \vio{} \vio{} \vio{**}
	\end{tableau}
\end{center}

\subsection{Syntax and Semantics}
I used forest over qtree because it's simpler and easier to use. However, both packages present a problem when writing semantics trees, since using square brackets already has another function. The \verb|\lb| command fixes this by placing brackets in math mode. In addition, the \verb|\lam| command outputs a math-mode \lam.
\begin{center}
\begin{forest}
	[S
		[NP [N [Alice]]]
		[VP
			[V [uses]]
			[NP [N [LaTeX]]]
		]
	]
\end{forest}
\qquad
\begin{forest}
	[\lb{A uses L}
		[NP [\lb{A} [Alice]]]
		[\lb{\lam y.y uses L}
			[\lb{\lam xy.y uses x} [uses]]
			[N [\lb{L} [LaTeX]]]
		]
	]
\end{forest}
\end{center}
Along with trees, there's support for semantic denotations, given by the \verb|\denot[]{}| command. For example, \verb|\denot[\alpha]{Alice}| yields \denot[\alpha]{Alice}. Denotations can also wrap larger objects. For trees, use the \verb|\denotree[]{}| command \cite{heim1998}
\begin{center}
\denotree{\begin{forest} [S [NP [N [Alice]]] [VP [V [runs]]]] \end{forest}} = \denot{runs}(\denot{Alice})
\end{center}
For logicians, the packages \texttt{bussproofs} and \texttt{frege} are also provided. The bussproofs package is for natural deduction and sequent proofs:
\begin{prooftree}
\Axiom$\Gamma, A, B\ \fCenter\ B$
\UnaryInf$\Gamma, A\ \fCenter\ (B \to C)$
\UnaryInf$\Gamma\ \fCenter\ (A \to (B \to C))$
\end{prooftree}
The frege package allows for typesetting \textit{Begriffsschrift}, like
\[\Fconditional[\Facontent]{\Fcontent \Fcontent A}{\Fcontent\Fconditional
{\Fcontent B}{\Fcontent C}}\]

\section{Natural Sciences}
The pre-loaded natural science packages are \texttt{siunitx}, \texttt{chemfig} and \texttt{mhchem} for chemistry, and \texttt{physics}, \texttt{tensor}, and \texttt{circuitikz} (with the \texttt{american} option) for physics.

Some non-SI units have been added, like \verb|\mile| and \verb|\atm|. If a unit isn't covered by the package or by \texttt{siunitx}, you can use the \verb|\unt{UNIT}| command, which writes the unit preceded by a space, and is ideal for boxing answers in math mode, like $\boxed{-3.14\unt{N}}$.

Science notes will usually not require the theorem environment, but the two following environments may be useful. The \verb|eqn| environment is a \verb|thm| class environment which can be used for any important equations, like
\begin{eqn}[Gibbs Free Energy \cite{gibbs1873}]
	The change in Gibbs Free Energy of a reaction with change in Enthalpy $\Delta H$, temperature $T$, and change in entropy $\Delta S$ is
	\[\Delta G\degr = \Delta H\degr - T\Delta S\degr.\]
\end{eqn}
The other useful environment is the \verb|law| environment, used for describing mathematical and scientific laws, such as
\begin{law}[Newton's Second Law]
	The net force on an object with mass $m$ is
	\[\Sigma F = ma,\]
	where $a$ is its acceleration \cite{newton1687}.
\end{law}
Sometimes, it may be helpful to box diagrams as well. This can be done with the \verb|diagram| environment (a \texttt{thm} class environment)
\begin{diagram}[Glucose]
The below diagram is a glucose molecule
\begin{center}
	\chemfig{[:-30]HO--[:30](<[2]OH)-(<:[6]OH)
-[:30](<:[2]OH)-(<:[6]OH)-[:30](=[2]O)-H}
\end{center}
\end{diagram}

\newpage

\section{Acknowledgements and References}
Much of my knowledge about how to create and use packages is from various Overleaf tutorials and \TeX{} StackExchange posts. I also learned about using mdframed and thmtools from Evan Chen's \texttt{evan.sty} package.

\bibliographystyle{plain}
\bibliography{classnotescite.bib}

\end{document}